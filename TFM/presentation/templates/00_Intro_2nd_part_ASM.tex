\documentclass[10pt,compress]{beamer}
\mode<presentation>
\usepackage{graphicx}
%\usepackage{beamerthemesplit}
\usepackage[ansinew]{inputenc}
%\usepackage[spanish]{babel}
\usepackage{amsmath,amssymb}
\usepackage{eqlist}
\usepackage{amsfonts} % Simbolos matematicos
\usepackage{amsthm} %Enunciados predefinidos y demostraciones
\usepackage{bbm}
\usepackage{enumerate}
\usepackage[dvips]{epsfig} %Incluir figuras .eps
\usepackage{epsf}%Incluir figuras .eps
\usepackage{eurosym}
\usepackage{subfigure}
%\usetheme{Malmoe} 


\usepackage{multirow}

\useoutertheme[footline=authortitle]{miniframes} % Sections at the top of slides
\useinnertheme{circles}

\usepackage{chicago}%,e:/latex/psfrag

\input{slides_English.sty}

\def\noframe#1{}
\newcommand{\mb}{\mbox}
\newcommand{\ts}{\thickspace}
\newcommand{\xq}{\overline{\mb{x}}}
\newcommand{\yq}{\overline{\mb{y}}}
\renewcommand{\baselinestretch}{1,2}
%\newcommand{\R}{ \mathbb{R} }

\def\red#1{\textcolor[rgb]{1.00,0.00,0.00}{#1}}
\def\green#1{\textcolor[rgb]{0.20,0.70,0.10}{#1}}
\def\blue#1{\textcolor[rgb]{0.00,0.00,1.00}{#1}}

%\def\citeN{\cite}    
\def\bi{\begin{itemize}}
\def\ei{\end{itemize}}
\def\F{\mathcal{F}}
\def\then{\Rightarrow}
\def\tiende{\longrightarrow}

\def\MSE{\mbox{MSE}}
\def\ISE{\mbox{ISE}}
\def\MISE{\mbox{MISE}}
\def\IMSE{\mbox{IMSE}}
\def\AMISE{\mbox{AMISE}}
\def\AMSE{\mbox{AMSE}}
\def\Sesgo{\mbox{Sesgo}}
\def\Bias{\mbox{Bias}}
\def\Var{\mbox{Var}}
\def\IQR{\mbox{IQR}}
\def\LSCV{\mbox{LSCV}}
\def\CV{\mbox{CV}}
\def\GCV{\mbox{GCV}}
\def\PI{\mbox{PI}}
\def\ECMP{\mbox{ECMP}}
\def\test{\mbox{\scriptsize test}}


\def\diag{\mbox{Diag}}
\def\Traza{\mbox{Traza}}
\def\traza{\mbox{Traza}}
\def\tr{^{\mbox{\scriptsize T}}}

%\newtheorem{teorema}{Teorema}

\def\comment#1{}
\def\practice#1{\frame{\Large {\bf \green{Practice:}} \\ #1}}

%%%%%%%%%%
%    1   %
%%%%%%%%%%
\title[ASM. Intro 2nd Part ASM, MIRI. \qquad{} \qquad{} \insertframenumber /\inserttotalframenumber]
{
%Advanced Statistical Modeling
%\\[1.5mm]
%Part 2. Nonparametric Modeling
%\\[2.5mm]
Introduction to the 2nd Part of ASM:
\\
{\bf 
Beyond LM and GLM
}
}
\author[Pedro Delicado]{{\Large Pedro Delicado}\\ \ \\
{\small Departament d'Estad�stica i Investigaci� Operativa \\
Universitat Polit�cnica de Catalunya}}
\date{}%November, December 2015}


\begin{document}

%%%%%%%%%%
%    2   %
%%%%%%%%%%

\frame{\titlepage}

%\section{}
%\frame{\scriptsize{\tableofcontents} }
%\AtBeginSubsection[]
%{
%  \begin{frame}<beamer>
%    \frametitle{}
%    \scriptsize{\tableofcontents[currentsection,currentsubsection]}
%  \end{frame}
%}
%
%\AtBeginSection[]
%{
%  \begin{frame}<beamer>
%   \frametitle{}
%    \scriptsize{\tableofcontents[currentsection,currentsubsection]}
%  \end{frame}
%}
%%%%%%%%%%%%%%%%%%%%%%%%%%%%%%%%%%%%%%%%%%%%%%%%%%%%%%%%%%%%%%

\frame{\frametitle{In the first part of Advanced Statistical Models}
\bi 
\item \blue{Multiple Linear Regression Model (LM):} 
\bi 
\item $Y=\bx\tr \bbeta +\varepsilon,\,\, Y\in\R, \bx\in\R^p$.
\item Data: $(Y_i,\bx_i)$, $i=1,\ldots,n$.
\item Estimation: Ordinary Least Squares (OLS),
\[
\hat{\bbeta}_{\mbox{\tiny OLS}} = \arg\min_{\bbeta} \sum_{i=1}^n \left(Y_i-\bx_i\tr \bbeta\right)^2.
\]
\ei 
\item \blue{Generalized Linear Model (LM):} 
\bi 
\item $(Y|\bX=\bx) \sim f(y;\mu(\bx)),\,\, Y\in S \subseteq \R, \bx\in\R^p$, \newline
$\mu(\bx)=\E(Y|\bX=\bx)$, $g(\mu(\bx))=\bx\tr \bbeta$, $\mu(\bx)=g^{-1}\left(\bx\tr \bbeta\right)$.
\item Data: $(Y_i,\bx_i)$, $i=1,\ldots,n$, independent.
\item Estimation: Maximum Likelihood,
\[
\hat{\bbeta}_{\mbox{\tiny ML}} = \arg\max_{\bbeta} 
\sum_{i=1}^n \log\left( f(Y_i;g^{-1}(\bx_i\tr \bbeta))\right).
\]
\ei 
\ei 
}

\frame{\frametitle{In the second part of Advanced Statistical Models}
\bi 
\item First, we explore \blue{penalized} estimation criteria:
\bi 
\item Ridge regression and Lasso for LM and for GLM.
\ei 
\item Second, we remove the assumptions of linearity. 
In fact we are not considering parametric forms for 
$m(\bx)=\E(Y|\bX=\bx)$ (\blue{nonparametric regression}):
\bi 
\item Local linear regression, local likelihood estimators.
\item Spline smoothing.
\item Generalized Additive Models (GAM).
\ei 
\ei 
}

\frame{\frametitle{Contents of 2nd part of the course}
	
	\vspace*{-.2cm}
	{\footnotesize
		\blue{\bf Regularized estimation of LM and GLM.}
		1. Ridge regression.
%		1.1. Linear estimators of a regression function.
%		1.2. Choosing the tunning parameter by CV and GCV.
		2. The Lasso estimation.
		3. Lasso estimation in the GLM.

		\blue{\bf Nonparametric regression model.}
		1. Local polynomial regression.
		2. Kernel functions.
		3. Theoretical properties. The bias-variance trade off.
		4. Bandwidth choice.
		
		\blue{\bf Generalized nonparametric regression model.}
		1. Nonparametric regression with binary response.
		2. Generalized nonparametric regression model.
		3. Estimation by maximum local likelihood. 
		
%		\blue{\bf Session 5: Inference with nonparametric regression.}
%		1. Variability bands.
%		2. Testing for no effects.
%		3. Checking a parametric model.
%		4. Comparing curves. 
%		
		\blue{\bf Spline smoothing.}
		1. Penalized least squares nonparametric regression.
		2. Splines, cubic splines and interpolation.
		3. Smoothing splines.
		4. B-splines and P-splines.
		5. Spline regression.
		6. Fitting generalized nonparametric regression models with splines.
		
		\blue{\bf Generalized additive models and Semiparametric models.}
		1. Multiple nonparametric regression. The curse of dimensionality.
		2. Additive models.
		3. Generalized additive models.
		4. Semiparametric models.
		
		%\blue{\bf Session 7}. Contents to be determined in a flexible way:
		%Buffer session; 
		%Introduction to Functional Data Analysis; ...
	}
}
\end{document}

%%%%%%%%%%%%%%%%%%%%%%%%%%%%%%%%%%%%%%%%%%%%%%%%%%%%%%%%%%%%%%%%%%%%%%%%%%%%%%%%%%%%%%%%%%%%%%%%%%%%%%%%%%%%%%%%%%%%%%%%%

%  Bibliograf�a
% Bibliograf�a utilizada para elaborar estas notas:
%\nocite{Silverman:1986}
%\nocite{Scott:1992}
\nocite{WanJon:1995}
\nocite{Simonoff:1996}
\nocite{FanGij:1996}
\nocite{BowAzz:1997}
\nocite{HasTibFri:2001}
\nocite{Wasserman:2006}
\nocite{Loader:1999}
\nocite{Wood:2006}

{\tiny 
\bibliographystyle{chicago}
%\bibliographystyle{e:/latex/chicago_esp}
%\bibliographystyle{g:/latex/chicago_esp}
%\bibliographystyle{c:/pedro/latex/chicago_esp}
\bibliography{ModNoPar}
}



\end{document}

